\section{Introduction}
The goal of this research effort was to compare the performance of DNS over multiple different protocols and ultimately compare them to HTTP/3.  The protocols used in the experiment were HTTP/1.0, HTTP/1.1, HTTP/2, HTTP/3, TLS1.0, TLS1.2, TLS1.3, and traditional DNS. These were chosen to give a broad scope of comparison and demonstrate the performance of popularly used protocols to the experimental HTTP/3. The HTTP/3 protocol is under consideration to become an internet standard, and is already being used by some popular internet companies such as Google.  There are also libraries that support the protocol in popular programming languages like Python, Rust, and C.  HTTP/3 boasts faster speeds and performance through the use of UDP and is also referred to as HTTP-over-QUIC where QUIC stands for “Quick UDP Internet Connections”.  Taking advantage of the already existing lower-level UDP protocol, HTTP/3 could be implemented without having to upgrade kernels of internet systems.  UDP is also not constrained by single data streams like TCP; one data stream failure does not have to impact the others.  The aim of the experiment was to gather timing data from various vantage points using multiple internet protocols resolving a DNS request.  The timing data was utilized for a comparison between the protocols.  It was found that HTTP/3 did not have any substantial performance improvements to the timings of the current protocols, but the difference could be miniscule when HTTP/3 might be effective for extremely poor internet connections or provide extra security.